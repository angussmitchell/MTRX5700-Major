\section{Background}

Whilst a dancing drone is not a completely new idea, they are poorly documented in academic literature, likely because of its limited usefulness outside of a fun and whimsical display of drone technology. We did, however, take to YouTube to find an abundance of dancing drone videos which we used as inspiration for what we could aim to achieve, the most impressive being Intel's swarm of 500 drones, a custom drone which Drexel University created for a dance company called Parsons Dance, and Parrot's own demonstration of the Bebop at a drone convention.\\

%% put these either as hyperlinks or as references
% insert hyperlink for Intel: https://www.youtube.com/watch?v=aOd4-T_p5fA
% insert hyperlink for Drexel: https://www.youtube.com/watch?v=SsPZZrXBUbY
% insert hyperlink for Parrot: https://www.youtube.com/watch?v=A_3UifFb45Y

If we split our dancing drone problem into a sound processing problem and a drone control problem, there is much more previous work which we can piggy-back off of, both in the academic space as well as in drone hobbyist culture.\\

ANGUS: please add paragraph or two about how the multiple methods of beat detection and music processing, e.g. thresholding of amplitude, framing, mfccs. you can also talk about how all that music processing stuff is common practice in the digital signal processing field of work.\\

NEILL: please write a few sentences about the Parrot SDK and maybe also mention two or three APIs, including the PS-drone library, for controlling the drone, and a sentence comparing using the SDK with using APIs.

Parrot provides a SDK for controlling the ARDrone.2, however this was only used as a reference. An alternative
implementation of drone control, the python library `PS-Drone' was used.
