\section{Background}

Whilst a dancing drone is not a completely new idea, they are poorly documented in academic literature, likely because of its limited usefulness outside of a fun and whimsical display of drone technology. We did, however, take to YouTube to find an abundance of dancing drone videos which we used as inspiration for what we could aim to achieve, the most impressive being Intel's swarm of 500 drones, a custom drone which Drexel University created for a dance company called Parsons Dance, and Parrot's own demonstration of the Bebop at a drone convention.\\

%% put these either as hyperlinks or as references
% insert hyperlink for Intel: https://www.youtube.com/watch?v=aOd4-T_p5fA
% insert hyperlink for Drexel: https://www.youtube.com/watch?v=SsPZZrXBUbY
% insert hyperlink for Parrot: https://www.youtube.com/watch?v=A_3UifFb45Y

If we split our dancing drone problem into a sound processing problem and a drone control problem, there is much more previous work which we can piggy-back off of, both in the academic space as well as in drone hobbyist culture.\\

Sound processing is central to creating an intersting and aesthetically pleasing dance. This involves finding a song's tempo or beat, and determining what type of dance move is appropriate. The main requirement for tempo detection is to determine when a beat is in the music, ultimately allowng the drone to time its dance moves. To determine which dance move to execute, a machine learning approach was taken. The objective of this is to split the song into several distinct sections, and to determine what dance is appropriate in each section. 



While parrot provides a SDK for controlling the ARDrone.2, however this was only used as a reference. An alternative
implementation of drone control, the python library `PS-Drone' was used.
